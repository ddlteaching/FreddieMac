%%
% Copyright (c) 2017, Pascal Wagler;  
% Copyright (c) 2014--2017, John MacFarlane
% 
% All rights reserved.
% 
% Redistribution and use in source and binary forms, with or without 
% modification, are permitted provided that the following conditions 
% are met:
% 
% - Redistributions of source code must retain the above copyright 
% notice, this list of conditions and the following disclaimer.
% 
% - Redistributions in binary form must reproduce the above copyright 
% notice, this list of conditions and the following disclaimer in the 
% documentation and/or other materials provided with the distribution.
% 
% - Neither the name of John MacFarlane nor the names of other 
% contributors may be used to endorse or promote products derived 
% from this software without specific prior written permission.
% 
% THIS SOFTWARE IS PROVIDED BY THE COPYRIGHT HOLDERS AND CONTRIBUTORS 
% "AS IS" AND ANY EXPRESS OR IMPLIED WARRANTIES, INCLUDING, BUT NOT 
% LIMITED TO, THE IMPLIED WARRANTIES OF MERCHANTABILITY AND FITNESS 
% FOR A PARTICULAR PURPOSE ARE DISCLAIMED. IN NO EVENT SHALL THE 
% COPYRIGHT OWNER OR CONTRIBUTORS BE LIABLE FOR ANY DIRECT, INDIRECT, 
% INCIDENTAL, SPECIAL, EXEMPLARY, OR CONSEQUENTIAL DAMAGES (INCLUDING,
% BUT NOT LIMITED TO, PROCUREMENT OF SUBSTITUTE GOODS OR SERVICES; 
% LOSS OF USE, DATA, OR PROFITS; OR BUSINESS INTERRUPTION) HOWEVER 
% CAUSED AND ON ANY THEORY OF LIABILITY, WHETHER IN CONTRACT, STRICT 
% LIABILITY, OR TORT (INCLUDING NEGLIGENCE OR OTHERWISE) ARISING IN 
% ANY WAY OUT OF THE USE OF THIS SOFTWARE, EVEN IF ADVISED OF THE 
% POSSIBILITY OF SUCH DAMAGE.
%%

%%
% For usage information and examples visit the GitHub page of this template:
% https://github.com/Wandmalfarbe/pandoc-latex-template
%%

\PassOptionsToPackage{unicode=true}{hyperref} % options for packages loaded elsewhere
\PassOptionsToPackage{hyphens}{url}
%
\documentclass[a4paper,]{scrartcl}
\usepackage{lmodern}
\usepackage{amssymb,amsmath}
\usepackage{ifxetex,ifluatex}
\usepackage{fixltx2e} % provides \textsubscript
\ifnum 0\ifxetex 1\fi\ifluatex 1\fi=0 % if pdftex
  \usepackage[T1]{fontenc}
  \usepackage[utf8]{inputenc}
  \usepackage{textcomp} % provides euro and other symbols
\else % if luatex or xelatex
  \usepackage{unicode-math}
  \defaultfontfeatures{Ligatures=TeX,Scale=MatchLowercase}
\fi
% use upquote if available, for straight quotes in verbatim environments
\IfFileExists{upquote.sty}{\usepackage{upquote}}{}
% use microtype if available
\IfFileExists{microtype.sty}{%
\usepackage[]{microtype}
\UseMicrotypeSet[protrusion]{basicmath} % disable protrusion for tt fonts
}{}
\IfFileExists{parskip.sty}{%
\usepackage{parskip}
}{% else
\setlength{\parindent}{0pt}
\setlength{\parskip}{6pt plus 2pt minus 1pt}
}
\usepackage{hyperref}
\hypersetup{
            pdftitle={Machine Learning with Python},
            pdfauthor={Abhijit DasGupta},
            pdfborder={0 0 0},
            breaklinks=true}
\urlstyle{same}  % don't use monospace font for urls
\usepackage[margin=2.5cm,includehead=true,includefoot=true,centering]{geometry}
\usepackage{color}
\usepackage{fancyvrb}
\newcommand{\VerbBar}{|}
\newcommand{\VERB}{\Verb[commandchars=\\\{\}]}
\DefineVerbatimEnvironment{Highlighting}{Verbatim}{commandchars=\\\{\}}
% Add ',fontsize=\small' for more characters per line
\newenvironment{Shaded}{}{}
\newcommand{\KeywordTok}[1]{\textcolor[rgb]{0.00,0.44,0.13}{\textbf{#1}}}
\newcommand{\DataTypeTok}[1]{\textcolor[rgb]{0.56,0.13,0.00}{#1}}
\newcommand{\DecValTok}[1]{\textcolor[rgb]{0.25,0.63,0.44}{#1}}
\newcommand{\BaseNTok}[1]{\textcolor[rgb]{0.25,0.63,0.44}{#1}}
\newcommand{\FloatTok}[1]{\textcolor[rgb]{0.25,0.63,0.44}{#1}}
\newcommand{\ConstantTok}[1]{\textcolor[rgb]{0.53,0.00,0.00}{#1}}
\newcommand{\CharTok}[1]{\textcolor[rgb]{0.25,0.44,0.63}{#1}}
\newcommand{\SpecialCharTok}[1]{\textcolor[rgb]{0.25,0.44,0.63}{#1}}
\newcommand{\StringTok}[1]{\textcolor[rgb]{0.25,0.44,0.63}{#1}}
\newcommand{\VerbatimStringTok}[1]{\textcolor[rgb]{0.25,0.44,0.63}{#1}}
\newcommand{\SpecialStringTok}[1]{\textcolor[rgb]{0.73,0.40,0.53}{#1}}
\newcommand{\ImportTok}[1]{#1}
\newcommand{\CommentTok}[1]{\textcolor[rgb]{0.38,0.63,0.69}{\textit{#1}}}
\newcommand{\DocumentationTok}[1]{\textcolor[rgb]{0.73,0.13,0.13}{\textit{#1}}}
\newcommand{\AnnotationTok}[1]{\textcolor[rgb]{0.38,0.63,0.69}{\textbf{\textit{#1}}}}
\newcommand{\CommentVarTok}[1]{\textcolor[rgb]{0.38,0.63,0.69}{\textbf{\textit{#1}}}}
\newcommand{\OtherTok}[1]{\textcolor[rgb]{0.00,0.44,0.13}{#1}}
\newcommand{\FunctionTok}[1]{\textcolor[rgb]{0.02,0.16,0.49}{#1}}
\newcommand{\VariableTok}[1]{\textcolor[rgb]{0.10,0.09,0.49}{#1}}
\newcommand{\ControlFlowTok}[1]{\textcolor[rgb]{0.00,0.44,0.13}{\textbf{#1}}}
\newcommand{\OperatorTok}[1]{\textcolor[rgb]{0.40,0.40,0.40}{#1}}
\newcommand{\BuiltInTok}[1]{#1}
\newcommand{\ExtensionTok}[1]{#1}
\newcommand{\PreprocessorTok}[1]{\textcolor[rgb]{0.74,0.48,0.00}{#1}}
\newcommand{\AttributeTok}[1]{\textcolor[rgb]{0.49,0.56,0.16}{#1}}
\newcommand{\RegionMarkerTok}[1]{#1}
\newcommand{\InformationTok}[1]{\textcolor[rgb]{0.38,0.63,0.69}{\textbf{\textit{#1}}}}
\newcommand{\WarningTok}[1]{\textcolor[rgb]{0.38,0.63,0.69}{\textbf{\textit{#1}}}}
\newcommand{\AlertTok}[1]{\textcolor[rgb]{1.00,0.00,0.00}{\textbf{#1}}}
\newcommand{\ErrorTok}[1]{\textcolor[rgb]{1.00,0.00,0.00}{\textbf{#1}}}
\newcommand{\NormalTok}[1]{#1}
\setlength{\emergencystretch}{3em}  % prevent overfull lines
\providecommand{\tightlist}{%
  \setlength{\itemsep}{0pt}\setlength{\parskip}{0pt}}
\setcounter{secnumdepth}{0}
% Redefines (sub)paragraphs to behave more like sections
\ifx\paragraph\undefined\else
\let\oldparagraph\paragraph
\renewcommand{\paragraph}[1]{\oldparagraph{#1}\mbox{}}
\fi
\ifx\subparagraph\undefined\else
\let\oldsubparagraph\subparagraph
\renewcommand{\subparagraph}[1]{\oldsubparagraph{#1}\mbox{}}
\fi

% Make use of float-package and set default placement for figures to H
\usepackage{float}
\floatplacement{figure}{H}


\title{Machine Learning with Python}
\author{Abhijit DasGupta}
\date{October 8, 2017}





%%
%% added
%%

%
% No language specified? take American English.
%

\ifnum 0\ifxetex 1\fi\ifluatex 1\fi=0 % if pdftex
  \usepackage[shorthands=off,main=english]{babel}
\else
  % load polyglossia as late as possible as it *could* call bidi if RTL lang (e.g. Hebrew or Arabic)
  \usepackage{polyglossia}
  \setmainlanguage[]{english}
\fi


%
% colors
%
\usepackage[dvipsnames,svgnames*,x11names*,table]{xcolor}

%
% listing colors
%
\definecolor{listing-background}{HTML}{F7F7F7}
\definecolor{listing-rule}{HTML}{B3B2B3}
\definecolor{listing-numbers}{HTML}{B3B2B3}
\definecolor{listing-text-color}{HTML}{000000}
\definecolor{listing-keyword}{HTML}{435489}
\definecolor{listing-identifier}{HTML}{435489}
\definecolor{listing-string}{HTML}{00999A}
\definecolor{listing-comment}{HTML}{8E8E8E}
\definecolor{listing-javadoc-comment}{HTML}{006CA9}

%\definecolor{listing-background}{rgb}{0.97,0.97,0.97}
%\definecolor{listing-rule}{HTML}{B3B2B3}
%\definecolor{listing-numbers}{HTML}{B3B2B3}
%\definecolor{listing-text-color}{HTML}{000000}
%\definecolor{listing-keyword}{HTML}{D8006B}
%\definecolor{listing-identifier}{HTML}{000000}
%\definecolor{listing-string}{HTML}{006CA9}
%\definecolor{listing-comment}{rgb}{0.25,0.5,0.35}
%\definecolor{listing-javadoc-comment}{HTML}{006CA9}

%
% for the background color of the title page
%

%
% TOC depth and 
% section numbering depth
%
\setcounter{tocdepth}{3}

%
% line spacing
%
\usepackage{setspace}
\setstretch{1.2}

%
% break urls
%
\PassOptionsToPackage{hyphens}{url}

%
% When using babel or polyglossia with biblatex, loading csquotes is recommended 
% to ensure that quoted texts are typeset according to the rules of your main language.
%
\usepackage{csquotes}

%
% captions
%
\usepackage[font={small,it}]{caption}
\newcommand{\imglabel}[1]{\textbf{\textit{(#1)}}}

%
% Source Sans Pro as the de­fault font fam­ily
% Source Code Pro for monospace text
%
% 'default' option sets the default 
% font family to Source Sans Pro, not \sfdefault.
%
\usepackage[default]{sourcesanspro}
\usepackage{sourcecodepro}

%
% heading font
%
\newcommand*{\heading}{\fontfamily{\sfdefault}\selectfont}

%
% heading color
%
\usepackage{relsize}
\usepackage{sectsty}
\definecolor{almostblack}{RGB}{40,40,40}
\allsectionsfont{\sffamily\color{almostblack}}

%
% variables for title and author
%
\usepackage{titling}
\title{Machine Learning with Python}
\author{Abhijit DasGupta}

%
% environment for boxes
%
%\usepackage{framed}

%
% tables
%

%
% remove paragraph indention
%
\setlength{\parindent}{0pt}
\setlength{\parskip}{6pt plus 2pt minus 1pt}
\setlength{\emergencystretch}{3em}  % prevent overfull lines

%
%
% Listings
%
%


%
% header and footer
%
\usepackage{fancyhdr}
\pagestyle{fancy}
\fancyhead{}
\fancyfoot{}
\lhead{Machine Learning with Python}
\chead{}
\rhead{October 8, 2017}
\lfoot{Abhijit DasGupta}
\cfoot{}
\rfoot{\thepage}
\renewcommand{\headrulewidth}{0.4pt}
\renewcommand{\footrulewidth}{0.4pt}

%%
%% end added
%%

\begin{document}

%%
%% begin titlepage
%%


%%
%% end titlepage
%%


{
\setcounter{tocdepth}{3}
\tableofcontents
}
\section{Introduction}\label{introduction}

Python is a popular general-purpose computing language. It is an
open-source language released under a liberal
\href{https://docs.python.org/3/license.html}{license} that is
compatible with the
\href{https://www.gnu.org/licenses/gpl-3.0.en.html}{GPL}.

In recent times, Python has become one of the preferred open-source
languages for doing data science (along with
\href{http://www.r-project.org}{R}). This has been driven by the
development of the \href{http://www.numpy.org}{\texttt{numpy}},
\href{http://www.scipy.org}{\texttt{scipy}} and
\href{http://matplotlib.org}{\texttt{matplotlib}} packages in the 90s to
mimic Matlab, and then development of
\href{http://pandas.pydata.org}{\texttt{pandas}},
\href{http://www.statsmodels.org}{\texttt{statsmodels}} and
\href{http://scikit-learn.org}{\texttt{sckit-learn}} in the 2000s to add
statistical and machine learning functionality akin to R. This has come
to be known, along with some other packages, as the
\href{https://pydata.org/downloads.html}{PyData Stack}.

\subsection{Installing Python}\label{installing-python}

The easiest way to install Python for data science is using the Anaconda
Python Distribution, provided by
\href{https://www.anaconda.com/}{Anaconda, Inc.}. This distribution
bundles together over 400 packages (depending on your operating system)
useful for data science applications. To install Python:

\begin{enumerate}
\def\labelenumi{\arabic{enumi}.}
\tightlist
\item
  Download the Anaconda Installer from
  \href{https://www.anaconda.com/download}{Anaconda} based on your
  operating system. Currently Python version 3.6 is preferred since the
  support for Python version 2.7 will cease soon.
\item
  Open the installer and install Anaconda
\end{enumerate}

\begin{quote}
Note for Mac users: The Mac OS comes with a default Python installation
that is part of the operating system. Anaconda is installed at a
different location and doesn't overwrite the system Python. The
installation changes the default Python to Anaconda, so when you run
Python from the terminal by typing \texttt{python}, the Anaconda version
will be used. Optionally you can keep your default system version of
Python as the default and create an alias in your .bashrc file to access
the Anaconda version of Python.
\end{quote}

\subsection{Training}\label{training}

This training will consist of four modules:

\begin{enumerate}
\def\labelenumi{\arabic{enumi}.}
\tightlist
\item
  \protect\hyperlink{IntroToPython}{Introduction to Python}
\item
  \protect\hyperlink{DecisionTrees}{Decision Trees}
\item
  \protect\hyperlink{RandomForests}{Bagging and Random Forests}
\item
  \protect\hyperlink{XGBoost}{Boosting and XGBoost}
\end{enumerate}

We will start with an introduction to Python programming for new users
of Python, to get users up to speed with basic Python syntax for data
science. This will lead up to using basic \texttt{pandas} for data
manipulation. Additional Python packages will be introduced in later
sections. We will introduce selected intermediate Python programming
concepts with an explanatory note, as needed.

Next, we will introduce
\href{https://en.wikipedia.org/wiki/Decision_tree_learning}{decision
trees}, specifically the Classification and Regression trees, or CART.
We will discuss the conceptual basis of decision trees with binary
splits. We will then formulate the algorithm and derive a Python program
to implement decision trees. Next, we will introduce the
\texttt{scikit-learn} package and its implementation of decision trees.
We will learn how to train decision trees, score new data to make
predictions, and tune decision trees for optimal performance.

Ensemble learning is a general method for using multiple learning
methods to derive a meta-machine that can perform better than the
original machines. We develop two ensemble machines using decision trees
as \emph{base learners}, namely, Random Forests and Gradient Boosted
Machines. The former is based on the general method of \emph{bagging} or
\emph{bootstrap aggregating} a number of base learners to create an
improved predictive engine. The latter uses an optimization principle
called \emph{boosting} which recursively fits base learners to data to
optimize some loss function or fit criterion. The particular
implementation of boosting that we will explore is
\href{http://xgboost.readthedocs.io/en/latest/}{\emph{XGBoost}}, a
scalable, fast and efficient implementation of gradient boosted trees.

As we proceed with these four modules, we will also introduce various
important statistical and computational concepts that are relevant to
machine learning, for example, the bias-variance tradeoff, prediction
error and its assessment in ensemble models, and others.

\hypertarget{IntroToPython}{\section{Introduction to
Python}\label{IntroToPython}}

\subsection{Base Python}\label{base-python}

Python works as a calculator

\begin{Shaded}
\begin{Highlighting}[]
\DecValTok{1}\OperatorTok{+}\DecValTok{1}
\end{Highlighting}
\end{Shaded}

\begin{verbatim}
2
\end{verbatim}

\begin{Shaded}
\begin{Highlighting}[]
\DecValTok{4}\OperatorTok{*}\DecValTok{6}
\end{Highlighting}
\end{Shaded}

\begin{verbatim}
24
\end{verbatim}

More generally, we can assign values to \emph{variables}. For example:

\begin{Shaded}
\begin{Highlighting}[]
\NormalTok{a }\OperatorTok{=} \StringTok{'123'}
\NormalTok{b }\OperatorTok{=} \DecValTok{123}
\NormalTok{c }\OperatorTok{=} \FloatTok{123.0}
\end{Highlighting}
\end{Shaded}

Each of a, b and c are of different \emph{types}:

\begin{Shaded}
\begin{Highlighting}[]
\BuiltInTok{print}\NormalTok{(}\BuiltInTok{type}\NormalTok{(a))}
\BuiltInTok{print}\NormalTok{(}\BuiltInTok{type}\NormalTok{(b))}
\BuiltInTok{print}\NormalTok{(}\BuiltInTok{type}\NormalTok{(c))}
\end{Highlighting}
\end{Shaded}

\begin{verbatim}
<class 'str'>
<class 'int'>
<class 'float'>
\end{verbatim}

\end{document}
