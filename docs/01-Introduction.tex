\documentclass[a4paper,11pt,final]{article}
        \usepackage{fancyvrb, color, graphicx, hyperref, amsmath, url, textcomp}
        \usepackage{palatino}
        \usepackage[a4paper,text={16.5cm,25.2cm},centering]{geometry}

        %Set different options for xetex and luatex
        \usepackage{iftex}
        \ifxetex\usepackage{fontspec}\fi

        \ifluatex\usepackage{fontspec}\fi

        \usepackage{xcolor}
        % ANSI colors from nbconvert
        \definecolor{ansi-black}{HTML}{3E424D}
        \definecolor{ansi-black-intense}{HTML}{282C36}
        \definecolor{ansi-red}{HTML}{E75C58}
        \definecolor{ansi-red-intense}{HTML}{B22B31}
        \definecolor{ansi-green}{HTML}{00A250}
        \definecolor{ansi-green-intense}{HTML}{007427}
        \definecolor{ansi-yellow}{HTML}{DDB62B}
        \definecolor{ansi-yellow-intense}{HTML}{B27D12}
        \definecolor{ansi-blue}{HTML}{208FFB}
        \definecolor{ansi-blue-intense}{HTML}{0065CA}
        \definecolor{ansi-magenta}{HTML}{D160C4}
        \definecolor{ansi-magenta-intense}{HTML}{A03196}
        \definecolor{ansi-cyan}{HTML}{60C6C8}
        \definecolor{ansi-cyan-intense}{HTML}{258F8F}
        \definecolor{ansi-white}{HTML}{C5C1B4}
         \definecolor{ansi-white-intense}{HTML}{A1A6B2}

        \hypersetup
        {   pdfauthor = {Pweave},
            pdftitle={Published from 01-Introduction.pmd},
            colorlinks=TRUE,
            linkcolor=black,
            citecolor=blue,
            urlcolor=blue
        }
        \setlength{\parindent}{0pt}
        \setlength{\parskip}{1.2ex}
        % fix for pandoc 1.14
        \providecommand{\tightlist}{%
            \setlength{\itemsep}{0pt}\setlength{\parskip}{0pt}}
        
\makeatletter
\def\PY@reset{\let\PY@it=\relax \let\PY@bf=\relax%
    \let\PY@ul=\relax \let\PY@tc=\relax%
    \let\PY@bc=\relax \let\PY@ff=\relax}
\def\PY@tok#1{\csname PY@tok@#1\endcsname}
\def\PY@toks#1+{\ifx\relax#1\empty\else%
    \PY@tok{#1}\expandafter\PY@toks\fi}
\def\PY@do#1{\PY@bc{\PY@tc{\PY@ul{%
    \PY@it{\PY@bf{\PY@ff{#1}}}}}}}
\def\PY#1#2{\PY@reset\PY@toks#1+\relax+\PY@do{#2}}

\expandafter\def\csname PY@tok@w\endcsname{\def\PY@tc##1{\textcolor[rgb]{0.73,0.73,0.73}{##1}}}
\expandafter\def\csname PY@tok@c\endcsname{\let\PY@it=\textit\def\PY@tc##1{\textcolor[rgb]{0.25,0.50,0.50}{##1}}}
\expandafter\def\csname PY@tok@cp\endcsname{\def\PY@tc##1{\textcolor[rgb]{0.74,0.48,0.00}{##1}}}
\expandafter\def\csname PY@tok@k\endcsname{\let\PY@bf=\textbf\def\PY@tc##1{\textcolor[rgb]{0.00,0.50,0.00}{##1}}}
\expandafter\def\csname PY@tok@kp\endcsname{\def\PY@tc##1{\textcolor[rgb]{0.00,0.50,0.00}{##1}}}
\expandafter\def\csname PY@tok@kt\endcsname{\def\PY@tc##1{\textcolor[rgb]{0.69,0.00,0.25}{##1}}}
\expandafter\def\csname PY@tok@o\endcsname{\def\PY@tc##1{\textcolor[rgb]{0.40,0.40,0.40}{##1}}}
\expandafter\def\csname PY@tok@ow\endcsname{\let\PY@bf=\textbf\def\PY@tc##1{\textcolor[rgb]{0.67,0.13,1.00}{##1}}}
\expandafter\def\csname PY@tok@nb\endcsname{\def\PY@tc##1{\textcolor[rgb]{0.00,0.50,0.00}{##1}}}
\expandafter\def\csname PY@tok@nf\endcsname{\def\PY@tc##1{\textcolor[rgb]{0.00,0.00,1.00}{##1}}}
\expandafter\def\csname PY@tok@nc\endcsname{\let\PY@bf=\textbf\def\PY@tc##1{\textcolor[rgb]{0.00,0.00,1.00}{##1}}}
\expandafter\def\csname PY@tok@nn\endcsname{\let\PY@bf=\textbf\def\PY@tc##1{\textcolor[rgb]{0.00,0.00,1.00}{##1}}}
\expandafter\def\csname PY@tok@ne\endcsname{\let\PY@bf=\textbf\def\PY@tc##1{\textcolor[rgb]{0.82,0.25,0.23}{##1}}}
\expandafter\def\csname PY@tok@nv\endcsname{\def\PY@tc##1{\textcolor[rgb]{0.10,0.09,0.49}{##1}}}
\expandafter\def\csname PY@tok@no\endcsname{\def\PY@tc##1{\textcolor[rgb]{0.53,0.00,0.00}{##1}}}
\expandafter\def\csname PY@tok@nl\endcsname{\def\PY@tc##1{\textcolor[rgb]{0.63,0.63,0.00}{##1}}}
\expandafter\def\csname PY@tok@ni\endcsname{\let\PY@bf=\textbf\def\PY@tc##1{\textcolor[rgb]{0.60,0.60,0.60}{##1}}}
\expandafter\def\csname PY@tok@na\endcsname{\def\PY@tc##1{\textcolor[rgb]{0.49,0.56,0.16}{##1}}}
\expandafter\def\csname PY@tok@nt\endcsname{\let\PY@bf=\textbf\def\PY@tc##1{\textcolor[rgb]{0.00,0.50,0.00}{##1}}}
\expandafter\def\csname PY@tok@nd\endcsname{\def\PY@tc##1{\textcolor[rgb]{0.67,0.13,1.00}{##1}}}
\expandafter\def\csname PY@tok@s\endcsname{\def\PY@tc##1{\textcolor[rgb]{0.73,0.13,0.13}{##1}}}
\expandafter\def\csname PY@tok@sd\endcsname{\let\PY@it=\textit\def\PY@tc##1{\textcolor[rgb]{0.73,0.13,0.13}{##1}}}
\expandafter\def\csname PY@tok@si\endcsname{\let\PY@bf=\textbf\def\PY@tc##1{\textcolor[rgb]{0.73,0.40,0.53}{##1}}}
\expandafter\def\csname PY@tok@se\endcsname{\let\PY@bf=\textbf\def\PY@tc##1{\textcolor[rgb]{0.73,0.40,0.13}{##1}}}
\expandafter\def\csname PY@tok@sr\endcsname{\def\PY@tc##1{\textcolor[rgb]{0.73,0.40,0.53}{##1}}}
\expandafter\def\csname PY@tok@ss\endcsname{\def\PY@tc##1{\textcolor[rgb]{0.10,0.09,0.49}{##1}}}
\expandafter\def\csname PY@tok@sx\endcsname{\def\PY@tc##1{\textcolor[rgb]{0.00,0.50,0.00}{##1}}}
\expandafter\def\csname PY@tok@m\endcsname{\def\PY@tc##1{\textcolor[rgb]{0.40,0.40,0.40}{##1}}}
\expandafter\def\csname PY@tok@gh\endcsname{\let\PY@bf=\textbf\def\PY@tc##1{\textcolor[rgb]{0.00,0.00,0.50}{##1}}}
\expandafter\def\csname PY@tok@gu\endcsname{\let\PY@bf=\textbf\def\PY@tc##1{\textcolor[rgb]{0.50,0.00,0.50}{##1}}}
\expandafter\def\csname PY@tok@gd\endcsname{\def\PY@tc##1{\textcolor[rgb]{0.63,0.00,0.00}{##1}}}
\expandafter\def\csname PY@tok@gi\endcsname{\def\PY@tc##1{\textcolor[rgb]{0.00,0.63,0.00}{##1}}}
\expandafter\def\csname PY@tok@gr\endcsname{\def\PY@tc##1{\textcolor[rgb]{1.00,0.00,0.00}{##1}}}
\expandafter\def\csname PY@tok@ge\endcsname{\let\PY@it=\textit}
\expandafter\def\csname PY@tok@gs\endcsname{\let\PY@bf=\textbf}
\expandafter\def\csname PY@tok@gp\endcsname{\let\PY@bf=\textbf\def\PY@tc##1{\textcolor[rgb]{0.00,0.00,0.50}{##1}}}
\expandafter\def\csname PY@tok@go\endcsname{\def\PY@tc##1{\textcolor[rgb]{0.53,0.53,0.53}{##1}}}
\expandafter\def\csname PY@tok@gt\endcsname{\def\PY@tc##1{\textcolor[rgb]{0.00,0.27,0.87}{##1}}}
\expandafter\def\csname PY@tok@err\endcsname{\def\PY@bc##1{\setlength{\fboxsep}{0pt}\fcolorbox[rgb]{1.00,0.00,0.00}{1,1,1}{\strut ##1}}}
\expandafter\def\csname PY@tok@kc\endcsname{\let\PY@bf=\textbf\def\PY@tc##1{\textcolor[rgb]{0.00,0.50,0.00}{##1}}}
\expandafter\def\csname PY@tok@kd\endcsname{\let\PY@bf=\textbf\def\PY@tc##1{\textcolor[rgb]{0.00,0.50,0.00}{##1}}}
\expandafter\def\csname PY@tok@kn\endcsname{\let\PY@bf=\textbf\def\PY@tc##1{\textcolor[rgb]{0.00,0.50,0.00}{##1}}}
\expandafter\def\csname PY@tok@kr\endcsname{\let\PY@bf=\textbf\def\PY@tc##1{\textcolor[rgb]{0.00,0.50,0.00}{##1}}}
\expandafter\def\csname PY@tok@bp\endcsname{\def\PY@tc##1{\textcolor[rgb]{0.00,0.50,0.00}{##1}}}
\expandafter\def\csname PY@tok@fm\endcsname{\def\PY@tc##1{\textcolor[rgb]{0.00,0.00,1.00}{##1}}}
\expandafter\def\csname PY@tok@vc\endcsname{\def\PY@tc##1{\textcolor[rgb]{0.10,0.09,0.49}{##1}}}
\expandafter\def\csname PY@tok@vg\endcsname{\def\PY@tc##1{\textcolor[rgb]{0.10,0.09,0.49}{##1}}}
\expandafter\def\csname PY@tok@vi\endcsname{\def\PY@tc##1{\textcolor[rgb]{0.10,0.09,0.49}{##1}}}
\expandafter\def\csname PY@tok@vm\endcsname{\def\PY@tc##1{\textcolor[rgb]{0.10,0.09,0.49}{##1}}}
\expandafter\def\csname PY@tok@sa\endcsname{\def\PY@tc##1{\textcolor[rgb]{0.73,0.13,0.13}{##1}}}
\expandafter\def\csname PY@tok@sb\endcsname{\def\PY@tc##1{\textcolor[rgb]{0.73,0.13,0.13}{##1}}}
\expandafter\def\csname PY@tok@sc\endcsname{\def\PY@tc##1{\textcolor[rgb]{0.73,0.13,0.13}{##1}}}
\expandafter\def\csname PY@tok@dl\endcsname{\def\PY@tc##1{\textcolor[rgb]{0.73,0.13,0.13}{##1}}}
\expandafter\def\csname PY@tok@s2\endcsname{\def\PY@tc##1{\textcolor[rgb]{0.73,0.13,0.13}{##1}}}
\expandafter\def\csname PY@tok@sh\endcsname{\def\PY@tc##1{\textcolor[rgb]{0.73,0.13,0.13}{##1}}}
\expandafter\def\csname PY@tok@s1\endcsname{\def\PY@tc##1{\textcolor[rgb]{0.73,0.13,0.13}{##1}}}
\expandafter\def\csname PY@tok@mb\endcsname{\def\PY@tc##1{\textcolor[rgb]{0.40,0.40,0.40}{##1}}}
\expandafter\def\csname PY@tok@mf\endcsname{\def\PY@tc##1{\textcolor[rgb]{0.40,0.40,0.40}{##1}}}
\expandafter\def\csname PY@tok@mh\endcsname{\def\PY@tc##1{\textcolor[rgb]{0.40,0.40,0.40}{##1}}}
\expandafter\def\csname PY@tok@mi\endcsname{\def\PY@tc##1{\textcolor[rgb]{0.40,0.40,0.40}{##1}}}
\expandafter\def\csname PY@tok@il\endcsname{\def\PY@tc##1{\textcolor[rgb]{0.40,0.40,0.40}{##1}}}
\expandafter\def\csname PY@tok@mo\endcsname{\def\PY@tc##1{\textcolor[rgb]{0.40,0.40,0.40}{##1}}}
\expandafter\def\csname PY@tok@ch\endcsname{\let\PY@it=\textit\def\PY@tc##1{\textcolor[rgb]{0.25,0.50,0.50}{##1}}}
\expandafter\def\csname PY@tok@cm\endcsname{\let\PY@it=\textit\def\PY@tc##1{\textcolor[rgb]{0.25,0.50,0.50}{##1}}}
\expandafter\def\csname PY@tok@cpf\endcsname{\let\PY@it=\textit\def\PY@tc##1{\textcolor[rgb]{0.25,0.50,0.50}{##1}}}
\expandafter\def\csname PY@tok@c1\endcsname{\let\PY@it=\textit\def\PY@tc##1{\textcolor[rgb]{0.25,0.50,0.50}{##1}}}
\expandafter\def\csname PY@tok@cs\endcsname{\let\PY@it=\textit\def\PY@tc##1{\textcolor[rgb]{0.25,0.50,0.50}{##1}}}

\def\PYZbs{\char`\\}
\def\PYZus{\char`\_}
\def\PYZob{\char`\{}
\def\PYZcb{\char`\}}
\def\PYZca{\char`\^}
\def\PYZam{\char`\&}
\def\PYZlt{\char`\<}
\def\PYZgt{\char`\>}
\def\PYZsh{\char`\#}
\def\PYZpc{\char`\%}
\def\PYZdl{\char`\$}
\def\PYZhy{\char`\-}
\def\PYZsq{\char`\'}
\def\PYZdq{\char`\"}
\def\PYZti{\char`\~}
% for compatibility with earlier versions
\def\PYZat{@}
\def\PYZlb{[}
\def\PYZrb{]}
\makeatother

        
\begin{document}
\section{Introduction}\label{introduction}

Python is a popular general-purpose computing language. It is an
open-source language released under a liberal
\href{https://docs.python.org/3/license.html}{license} that is
compatible with the
\href{https://www.gnu.org/licenses/gpl-3.0.en.html}{GPL}.

In recent times, Python has become one of the preferred open-source
languages for doing data science (along with
\href{http://www.r-project.org}{R}). This has been driven by the
development of the \href{http://www.numpy.org}{\texttt{numpy}},
\href{http://www.scipy.org}{\texttt{scipy}} and
\href{http://matplotlib.org}{\texttt{matplotlib}} packages in the 90s to
mimic Matlab, and then development of
\href{http://pandas.pydata.org}{\texttt{pandas}},
\href{http://www.statsmodels.org}{\texttt{statsmodels}} and
\href{http://scikit-learn.org}{\texttt{sckit-learn}} in the 2000s to add
statistical and machine learning functionality akin to R. This has come
to be known, along with some other packages, as the
\href{https://pydata.org/downloads.html}{PyData Stack}.

\subsection{Installing Python}\label{installing-python}

The easiest way to install Python for data science is using the Anaconda
Python Distribution, provided by
\href{https://www.anaconda.com/}{Anaconda, Inc.}. This distribution
bundles together over 400 packages (depending on your operating system)
useful for data science applications. To install Python:

\begin{enumerate}
\def\labelenumi{\arabic{enumi}.}
\tightlist
\item
  Download the Anaconda Installer from
  \href{https://www.anaconda.com/download}{Anaconda} based on your
  operating system. Currently Python version 3.6 is preferred since the
  support for Python version 2.7 will cease soon.
\item
  Open the installer and install Anaconda
\end{enumerate}

\begin{quote}
Note for Mac users: The Mac OS comes with a default Python installation
that is part of the operating system. Anaconda is installed at a
different location and doesn't overwrite the system Python. The
installation changes the default Python to Anaconda, so when you run
Python from the terminal by typing \texttt{python}, the Anaconda version
will be used. Optionally you can keep your default system version of
Python as the default and create an alias in your .bashrc file to access
the Anaconda version of Python.
\end{quote}

\subsection{Training}\label{training}

This training will consist of four modules:

\begin{enumerate}
\def\labelenumi{\arabic{enumi}.}
\tightlist
\item
  \protect\hyperlink{IntroToPython}{Introduction to Python}
\item
  \protect\hyperlink{DecisionTrees}{Decision Trees}
\item
  \protect\hyperlink{RandomForests}{Bagging and Random Forests}
\item
  \protect\hyperlink{XGBoost}{Boosting and XGBoost}
\end{enumerate}

We will start with an introduction to Python programming for new users
of Python, to get users up to speed with basic Python syntax for data
science. This will lead up to using basic \texttt{pandas} for data
manipulation. Additional Python packages will be introduced in later
sections. We will introduce selected intermediate Python programming
concepts with an explanatory note, as needed.

Next, we will introduce
\href{https://en.wikipedia.org/wiki/Decision_tree_learning}{decision
trees}, specifically the Classification and Regression trees, or CART.
We will discuss the conceptual basis of decision trees with binary
splits. We will then formulate the algorithm and derive a Python program
to implement decision trees. Next, we will introduce the
\texttt{scikit-learn} package and its implementation of decision trees.
We will learn how to train decision trees, score new data to make
predictions, and tune decision trees for optimal performance.

Ensemble learning is a general method for using multiple learning
methods to derive a meta-machine that can perform better than the
original machines. We develop two ensemble machines using decision trees
as \emph{base learners}, namely, Random Forests and Gradient Boosted
Machines. The former is based on the general method of \emph{bagging} or
\emph{bootstrap aggregating} a number of base learners to create an
improved predictive engine. The latter uses an optimization principle
called \emph{boosting} which recursively fits base learners to data to
optimize some loss function or fit criterion. The particular
implementation of boosting that we will explore is
\href{http://xgboost.readthedocs.io/en/latest/}{\emph{XGBoost}}, a
scalable, fast and efficient implementation of gradient boosted trees.

As we proceed with these four modules, we will also introduce various
important statistical and computational concepts that are relevant to
machine learning, for example, the bias-variance tradeoff, prediction
error and its assessment in ensemble models, and others.

\newpage
\end{document}